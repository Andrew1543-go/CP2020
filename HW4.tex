\documentclass[prb,papersize=a4paper,notitlepage]{revtex4-1}%
\usepackage{hyperref}
\usepackage{enumitem}
\usepackage{nicefrac}
\usepackage{amsmath}
\usepackage{graphicx}
\usepackage{amsfonts}
\usepackage{physics}
\usepackage{amssymb}
\usepackage{bm}
\usepackage[utf8]{inputenc}
\usepackage[russian]{babel}
\usepackage{listings}


\begin{document}

\title{Вычислительная физика, Осень 2020 ВШЭ. Задание 4.\footnote{Дополнительно указаны: (количество баллов за задачу)[имя задачи на nbgrader]}}
\maketitle
\begin{enumerate}
\item \textbf{(10)} Сгенерируйте случайную симметричную матрицу $A$ размера $1000 \times 1000$:
\lstset{language=Python}
\lstset{frame=lines}
\lstset{label={lst:code_direct}}
\lstset{basicstyle=\ttfamily}
\begin{lstlisting}
n = 1000
a = np.random.normal(0, 1, (n, n))
A = a + a.T
\end{lstlisting}
Используя функцию \lstinline{numpy.linalg.eigvalsh}, найдите спектр матрицы $A$ и постройте гистограмму распределения собственных значений (\lstinline{plt.hist}).
\item \textbf{(15)} Сгенерируйте случайную симметричную и положительно определенную матрицу $A$ размера $10 \times 10$:
\lstset{language=Python}
\lstset{frame=lines}
\lstset{label={lst:code_direct}}
\lstset{basicstyle=\ttfamily}
\begin{lstlisting}
n = 10
a = np.random.normal(0, 1, (n, n))
A = a @ a.T
\end{lstlisting}
Найдите спектр этой матрицы с помощью функции \lstinline{numpy.linalg.eigvalsh}. Имплементируйте $\textrm{QR}$--алгоритм без сдвигов (используйте \lstinline{numpy.linalg.qr}) и найдите спектр матрицы с его помощью. Сколько итераций требуется, чтобы приблизить минимальное собственное значение с точностью $1\%$?
\item \textbf{(15)} Рассмотрите матрицу 
$$
A=\begin{bmatrix}
0 & 1\\
0 & 0\\
\end{bmatrix}.\quad
$$
Найдите спектр матрицы $A$. Пусть $\sigma_\epsilon$ --множество таких [комплексных] $z$, что $z$ является собственным значением матрицы $A+\delta A$ c некоторым $||\delta A||_2<\epsilon$. Изобразите $\sigma_{0.1}$ и $\sigma_{0.01}$ (используйте без доказательства эквивалентность утверждений i и iv из задачи 26.1 Trefethen, Bau).
\item \textbf{(20)} Рассмотрите матрицу 
$$
A=\begin{bmatrix}
3 & 1 & 0 & 0\\
1 & 2 & 0 & 1\\
0 & 0 & 1 & 1\\
0 & 1 & 1 & 1
\end{bmatrix}\quad
$$
Реализуйте следующие методы нахождения максимального собственного значения (стартуйте со случайного вектора):
\begin{itemize}
\item Степенная итерация
\item Обратная итерация с $\mu = 3.5$
\item Обратная итерация с $\mu = 3.7$
\end{itemize}
Сколько шагов $k$ требуется в каждом случае для того, чтобы получить настоящий собственный вектор $v$ с точностью $||v-v_k||_2 < 10^{-3}$?
\item \textbf{(20)} Рассмотрите матрицу $A$ размера $32\times 32$, задаваемую следующей формулой:
$$
A_{ij} = -\delta_{i,j} + \delta_{i, j-1} + \delta_{i, j-2}.
$$
\begin{itemize}
\item Найдите спектр матрицы $A$.
\item Используя функцию \lstinline{scipy.linalg.expm}, постройте $||e^{At}||_2$ как функцию $t$ на интервале $0\le t \le 50$.
\item Используя (без доказательства) эквивалентность утверждений i и iii из задачи 26.1 Trefethen, Bau, изобразите в комплексной плоскости множество $\sigma_\epsilon$ [см. определение в Задаче 3] для $\epsilon = 10^{-i},\quad i=1,..,5$.
\end{itemize}
\item \textbf{(25)} Рассмотрите диагональную матрицу $D$ размера $n\times n$ и вектор--столбец $u$. Выберите $D$ и $u$ случайным образом (сгенерировав их элементы их стандартного нормального распределения) и найдите минимальное собственное значение и соответствующий собственный ему собственный вектор матрицы
$$A = D + \frac{u u^T}{u^T u}.$$ Рассмотрите случаи $n=10^2$ и $n=10^5$ -- во втором случае Вам, возможно, пригодится \href{https://en.wikipedia.org/wiki/Bunch%E2%80%93Nielsen%E2%80%93Sorensen_formula}{Bunch--Nielsen--Sorensen formula}
\end{enumerate}
\end{document}